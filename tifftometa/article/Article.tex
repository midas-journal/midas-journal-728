%
% Complete documentation on the extended LaTeX markup used for Insight
% documentation is available in ``Documenting Insight'', which is part
% of the standard documentation for Insight.  It may be found online
% at:
%
%     http://www.itk.org/

\documentclass{InsightArticle}


%%%%%%%%%%%%%%%%%%%%%%%%%%%%%%%%%%%%%%%%%%%%%%%%%%%%%%%%%%%%%%%%%%
%
%  hyperref should be the last package to be loaded.
%
%%%%%%%%%%%%%%%%%%%%%%%%%%%%%%%%%%%%%%%%%%%%%%%%%%%%%%%%%%%%%%%%%%
\usepackage[dvips,
bookmarks,
bookmarksopen,
backref,
colorlinks,linkcolor={blue},citecolor={blue},urlcolor={blue},
]{hyperref}
% to be able to use options in graphics
\usepackage{graphicx}
% for pseudo code
\usepackage{listings}
% subfigures
\usepackage{subfigure}


%  This is a template for Papers to the Insight Journal. 
%  It is comparable to a technical report format.

% The title should be descriptive enough for people to be able to find
% the relevant document. 
\title{Including TIFF tags in the MetaDataDictionary}

% Increment the release number whenever significant changes are made.
% The author and/or editor can define 'significant' however they like.
\release{0.00}

% At minimum, give your name and an email address.  You can include a
% snail-mail address if you like.
\author{Richard Beare}
\authoraddress{Richard.Beare@med.monash.edu.au}

\begin{document}
\maketitle

\ifhtml
\chapter*{Front Matter\label{front}}
\fi


\begin{abstract}
\noindent
% The abstract should be a paragraph or two long, and describe the
% scope of the document.
Tiff images are widely used and domain and instrument specific
information is often included in some tags. There isn't any standard
way of making use of this information in ITK at present. This article
introduces some minor additions to the ITK tiff reader to include the
content of some potentially useful tags in the image
MetaDataDictionary, thus enabling domain specific information to be
extracted from the tags within an ITK application.
\end{abstract}

\tableofcontents

\section{Introduction}
Tiff is a widely used image format that is frequently extended to new
domains by including information in fields in the tiff directory. The
directory in tiff files is equivalent to the header in other formats,
but needn't be at the start of the file, and multi image tiff file
will have multiple directories.

Most standard image information describing image pixel layout on disk,
image width, image height, colour encoding and so on is encoded in
``standard'' tiff fields and is already used by the ITK image
reader. However additional information that may be useful to some
applications can be stored in other tiff fields. This article
describes how a modified tiff reader can make these fields available
to an ITK application via the MetaDataDictionary.

\section{Domain specific extensions}
An example of domain specific extensions to tiff is the Open
Microscopy Environment (OME) tiff format --
\url{http://www.ome-xml.org/wiki/OmeTiff}. This extension encodes a
large range of microscopy related meta-data in XML in the tiff {\em
  ImageDescription} field. 

Correct interpretation of the image by an application may require
access to information in this field. In many cases a customized reader
may be more appropriate, but this is not always feasible in the early
days of using a new format. The extensions described in this article
provide an intermediate solution.

\section{TIFF fields}
The fields likely to be relevant to an application include:
\begin{itemize}
\item ARTIST
\item COPYRIGHT
\item DATETIME
\item DOCUMENTNAME
\item HOSTCOMPUTER
\item IMAGEDESCRIPTION
\item INKNAMES
\item MAKE
\item MODEL
\item PAGENAME
\item SOFTWARE
\item TARGETPRINTER
\item XMLPACKET
\end{itemize}

The {\em tiffinfo} tool distributed with libtiff provides a dump of
these fields that is useful when interpreting a new tiff variant. The
code in this article makes the same information available to an ITK
application.

\section{TIFFImageIO modifications}
Modifications to TIFFImageIO have been made and the files are included
in the source package. The two files, {\em itkTIFFImageIO.h} and {\em
  itkTIFFImageIO.cxx} are replacements for existing versions in {\em
  InsightToolkit/Code/IO}. The examples following assume that ITK has
been built using these files.

The modifications simply encode the field names and contents in the MetaDataDictionary.

\section{Accessing the fields via the MetaDataDictionary}
Standard mechanisms can be used to access the contents of the
MetaDataDictionary, as illustrated below. This example is from {\em check.cxx}, included
in this package. 

\small \begin{verbatim}
int main(int argc, char * argv[])
{

  if( argc != 2 )
    {
    std::cerr << "usage: " << argv[0] << " intput " << std::endl;
    std::cerr << " input: the input image" << std::endl;
    // std::cerr << "  : " << std::endl;
    exit(1);
    }

  const int dim = 2;
  
  typedef unsigned char PType;
  typedef itk::Image< PType, dim > IType;


  typedef itk::ImageFileReader< IType > ReaderType;
  ReaderType::Pointer reader = ReaderType::New();
  reader->SetFileName( argv[1] );
  reader->Update();

  typedef itk::MetaDataDictionary   DictionaryType;
  typedef itk::MetaDataObject< std::string > MetaDataStringType;

  const  DictionaryType & dictionary = reader->GetImageIO()->GetMetaDataDictionary();
  DictionaryType::ConstIterator itr = dictionary.Begin();
  DictionaryType::ConstIterator end = dictionary.End();
  while( itr != end )
    {
    itk::MetaDataObjectBase::Pointer  entry = itr->second;

    MetaDataStringType::Pointer entryvalue = 
      dynamic_cast<MetaDataStringType *>( entry.GetPointer() );
    if( entryvalue )
      {
      std::string tagkey   = itr->first;
      std::cout << "MetaDataDict " << tagkey << ": " << entryvalue->GetMetaDataObjectValue() << std::endl;

      }

    ++itr;
    }

  return 0;
}
\end{verbatim} \normalsize

\section{Sample output}
Apply check to an OME tiff included in the package provides the following output (line breaks included for clarity):

\small \begin{verbatim} 
 > check images/single-channel.ome.tif

 MetaDataDict TIFFTAG_IMAGEDESCRIPTION: <?xml version="1.0"
 encoding="UTF-8"?><!-- Warning: this comment is an OME-XML metadata
 block, which contains crucial dimensional parameters and other
 important metadata. Please edit cautiously (if at all), and back up
 the original data before doing so. For more information, see the
 OME-TIFF web site: http://loci.wisc.edu/ome/ome-tiff.html. --><OME
 xmlns="http://www.openmicroscopy.org/XMLschemas/OME/FC/ome.xsd"
 xmlns:xsi="http://www.w3.org/2001/XMLSchema-instance"
 xsi:schemaLocation="http://www.openmicroscopy.org/XMLschemas/OME/FC/ome.xsd
 http://www.openmicroscopy.org/XMLschemas/OME/FC/ome.xsd"><Image
 ID="openmicroscopy.org:Image:1" Name="single-channel"
 DefaultPixels="openmicroscopy.org:Pixels:1-1"><CreationDate>2007-11-08T14:52:24</CreationDate><Pixels
 ID="openmicroscopy.org:Pixels:1-1" DimensionOrder="XYZCT"
 PixelType="Uint8" BigEndian="true" SizeX="439" SizeY="167" SizeZ="1"
 SizeC="1" SizeT="1"><TiffData/></Pixels></Image></OME>

\end{verbatim} \normalsize

\section{Conclusions and further work}
The changes described in this article only provide access to the
standard fields in which data is stored as text. It is also possible
for vendors to include proprietary tiff fields, and future versions
may be able to include this information in the MetaDataDictionary.

However the current version should allow useful, domain specific,
interpretation of data without major changes to ITK infrastructure and
thus be useful in early project development. In particular,
information such as voxel size, scaling parameters and anything else
that might be relevant to correct image interpretation and encoded as
strings in tiff tags may be accessed in an ITK application via
relatively simple string parsing code.

\appendix



\bibliographystyle{plain}
\bibliography{InsightJournal}
\nocite{ITKSoftwareGuide}

\end{document}

